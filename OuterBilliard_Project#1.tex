\documentclass[11pt,a4paper]{article}

\usepackage{amsmath,amssymb}
\usepackage{graphicx}
\usepackage{epsfig}
\usepackage{url}



\begin{document}

\title{3 Outer Billiard}
\author{Marcel Vosshans}
\maketitle

\begin{abstract}
In the following report, we analyse the trajectory of a billiard ball if the ball tangents a corner of an equilateral triangle and moves the same distance forward, it needed to the corner. The question is: Can the ball escape? A Mathematica model simulates this model with the result, that the ball can’t escape from the triangle. This is a simplified model about our solar system and the possibility of an escaping planet based on a billiard table.
\end{abstract}

\tableofcontents

\section{Introduction}
Imagine a billiard table with a triangle on it. A billiard ball runs toward to one of the corners with constant speed v1. The way d1 it needs to the corner, is also the distance it runs forward (from the corner away). Then the ball targets the next corner and got a new distance to the corner: d2 which is also the new distance from the corner away. This procedure it makes n times. The total distance amount to:



The question is: Can the billiard ball escape from the triangle (divergent) or is there a border to reach which would be the local maximum circle around the triangle (convergent). This theory describes a mathematical model of the solar system an the possibility of an escaping planet (for example the Pluto [currently a dwarf planet]).

	
 	\subsection{Setup for the Model}


  	\subsubsection{A subheading}



\section{Results}
>>TODO<<


\section{Discussion}
>>TODO<<

\begin{thebibliography}{99}

 \bibitem{doc}Wolfram Language \& System, \url{http://reference.wolfram.com/language/}, 2017.09.20.


\end{thebibliography}

\end{document}
